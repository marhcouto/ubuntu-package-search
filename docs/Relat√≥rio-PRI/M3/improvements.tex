\section{Improvements}

We applied two strategies in order to try to improve our system:

\begin{itemize}
    \item application of independent boosts based on the number of downloads of a certain package (and other similar data)
    \item application of a relevance feedback algorithm; we chose Rocchio
\end{itemize}

In order to make possible one of these strategies, as well as increasing the possibilities of queries made to the system, a GUI was created.

\subsection{Independent Boosts}

Our data set provides a list of dependencies for almost all packages so we thought of using the number of packages that depend om some package as a good indicator of package relevance, but this strategy highly benefits libraries to the detriment of programs.

Another good indicator of a package relevance is the number of downloads a certain package has. It works like a reputation system: the most downloaded packages have more probability of being relevant. Better even would be a signal that would indicate how much users use a certain package. However, aptitude does not provide any of this data, so we restored to use statistics provided by Debian Popularity Contest \footnote{\url{https://popcon.debian.org/}}. \emph{Popcon} collects user information about the number of debian package downloads, updates and regular users. In order to collect data about users they must install a package called \textit{popularity-contest}. Although we are building an Ubuntu package search systems most Ubuntu packages are shared between both Linux Distributions. For the data set we used, 207276 submissions were considered \cite{popcontest}.

From the signals provided by the data set, we utilized 2 of them to create a relevance formula:

\begin{description}
    \item[inst] number of people who installed this package
    \item[vote] number of people who use this package regularly
\end{description}

\textbf{Formula:} $inst*0.0001 + vote*0.01$

This data was added to the final data set through the addition of some instructions in the data processing pipeline.

\subsection{Relevance Feedback}

In the effort to further improve our system, we implemented a relevance feedback system with the help of \emph{Rocchio}. This algorithm was developed using the Vector Space Model\footnote{\url{https://en.wikipedia.org/wiki/Vector_space_model}} as its basis. This means that our query and documents are represented as vectors, the goal being to produce a query vector that maximizes the proximity with the \emph{centroid} of relevant documents while minimizing the proximity with the \emph{centroid} of irrelevant documents. 
The ideal query can then be translated as \cite{rocchioformulations}:
\begin{align}
    \overrightarrow{q}_{opt} = \frac{1}{|C_r|} \sum_{\overrightarrow{d_j} \in C_r} \overrightarrow{d_j} - \frac{1}{|C_{nr}|} \sum_{\overrightarrow{d_j} \in C_{nr}} \overrightarrow{d_j}
\end{align}

Where $\overrightarrow{q}_{opt}$ is the optimal query, $|C_r|$ is the collection of relevant documents and $|C_{nr}|$ is the collection of non-relevant documents. The issue with this formulation is that it requires the user to evaluate the whole collection. So we need to use the following approximation \cite{rocchioformulations}: 

\begin{align}
    \overrightarrow{q}_i = \alpha \overrightarrow{q}_0 + \beta \frac{1}{|D_r|} \sum_{\overrightarrow{d_j} \in D_r} \overrightarrow{d_j} - \gamma \frac{1}{|D_{nr}|} \sum_{\overrightarrow{d_j} \in D_{nr}} \overrightarrow{d_j}
\end{align}

Where $\overrightarrow{q}_i$ is the improved query and $\overrightarrow{q}_0$ is the initial query. $|D_r|$ and $|D_{nr}|$ are respectively the documents the user considered relevant and irrelevant. And finally $\alpha$, $\beta$ and $\gamma$ are weights assigned to each term. \cite{rocchioformulations}

\subsubsection{Implementation Details}

To implement the Rocchio algorithm we need to get information about three variables:

\begin{itemize}
    \item A vector with both document classes terms
    \item A vector with query terms
    \item List of relevant and irrelevant documents for a query
\end{itemize}

We implemented Rocchio only considering the field we called Description and to get document classes terms we used SOLR Term Vectors\footnote{\url{https://solr.apache.org/guide/8_0/the-term-vector-component.html}} that returned for each field term it's frequency.

To get the query terms we needed to turn on the query debug mode so that we could parse the data from the field \textit{parsedquery} that contains the functions that will be applied to each term thus obtaining them.

Finally, to obtain a list of relevant documents, we built a GUI that allows a user to mark relevant documents for a query and a backend that supports the \emph{Rocchio} algorithm.


\subsection{GUI}

The last thing made in Milestone 3 was the a Graphical User Interface (frontend). The frontend was developed using \emph{React.js} and aims to, not only make the \emph{Rocchio} application possible, but also provide a platform for usage of our system where filtering and weighting parameters can be passed to the search task.

In Figure \ref{fig:frontend-homepage1} we can see he homepage of the web-app. This page allows he user to perform a search. This search can be done with no filters and the default weight settings for each field, or with personalization of these aspects, which can be seen in Figure \ref{fig:frontend-homepage3}. The results page shows the list of packages resultant and all the information regarding them (more information can be attained by clicking 'see more' button) (Figure \ref{fig:frontend-resultspage2}. In the same figure, the buttons that allow for selection of the relevant documents are also present. Utilizing the 'requery' button, the previous query can be remade using the \emph{Rocchio} algorithm (Figure \ref{fig:frontend-resultspage3}). 



\begin{figure*}[]
    \centering
    \includegraphics[width=\textwidth]{resources/frontend/homepage3.png}
    \caption{Home page - Search with weights and filters}
    \label{fig:frontend-homepage3}
\end{figure*}




