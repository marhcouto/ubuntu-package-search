\section{Introduction}
\label{sec:Introduction}

This project is developed in the context of the Information Processing and Retrieval course, with the principal intent being to build an information retrieval system. In order to achieve this, the data on which the system will function on must be chosen, collected and prepared.   

\subsection{Topic}

Our intent was to make our project as relevant as possible. For this reason, we intended to choose a topic that would resonate with the community's needs. Looking into our own experience with technology, the internet and information systems, we found software packages a great candidate. Taking into account the difficulties most of us often have with finding the correct name for a software package or the right software for our needs without using a broader search system like Google, there is a certain demand for a information retrieval system of this kind. 

We chose to use \emph{Ubuntu}\footnote{\url{https://ubuntu.com/}}'s software packages specifically for two reasons:

\begin{itemize}
    \item \emph{Ubuntu} is one of the most utilized \emph{Linux} distribution worldwide
    \item \emph{Ubuntu}'s default packages are organized into 4 main repositories
\end{itemize}

These factors made it more feasible to find the data we needed for the project.



