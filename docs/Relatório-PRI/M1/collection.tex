\section{Collection}
\label{sec:Collection}

\subsection{Data Sources}

The package's names were firstly collected from the 4 main \emph{Ubuntu}'s repositories, from the official website\footnote{\url{http://archive.ubuntu.com/ubuntu/dists/jammy/}}, guaranteeing the integrity and legitimacy of the data. The packages are divided into 4 repositories:

\begin{enumerate}
    \item \textbf{Main: } Canonical-supported free and open-source software (6090 Packages)
    
    \item \textbf{Universe: } Community-maintained free and open-source software (59969 Packages)
    
    \item \textbf{Restricted: } Proprietary drivers for devices (686 Packages)
    
     \item \textbf{Multiverse: } Software restricted by copyright or legal issues (881 Packages)
\end{enumerate}

The rest of the information was collected from \emph{Aptitude}\footnote{\url{https://www.debian.org/doc/manuals/aptitude/rn01re01.en.html}}. Aptitude is a command line high level interface for the package manager.

\subsection{Collection Process}

In order to obtain the list of packages from each repository, we downloaded a compressed file with data from each package\cite{debctrlpackage}. 

When extracted, these files had data in a format that resembled a Debian control file\cite{ubuntuarchive}, but had fields that were not contemplated in Debian’s definition. After extraction we got the package names from these files using a script and stored them in a file where each line contained a package’s name. 

As a last step we then downloaded each package's information from Aptitude because it provided better descriptions that were needed for the next steps and parsed it to a CSV file (Comma Separated Value) with the help of a parser built by ourselves.

After collection we had a total of 67626 records that occupied 38.7 MiB. It is worth noting that we kept our dataset separated by repository until the end of the processing phase.

These steps are illustrated in Diagram  \ref{fig:pipeline}.

\begin{figure*}[]
\centering
  \includegraphics[width=0.8\textwidth]{resources/pipeline_diagram.pdf}
  \label{fig:pipeline}
  \caption{Pipeline}
\end{figure*}

\subsection{Makefile}

To help with data collection we prepared two tasks to be performed in our \textit{makefile}:

\begin{description}
    \item[fetch\_raw\_package\_lists] This task, as the name suggests, downloads package's information from the \emph{Ubuntu} archives and places them inside a folder called \textit{raw\_data}.
    
    \item[build\_package\_lists] This task is used to scrape the package name out of \emph{Ubuntu}'s package descriptions that were downloaded in the previous step. The results are placed into a folder called \textit{package\_lists}
    
    \item[fetch\_packages] This task downloads package data for each repository from Aptitude into a CSV file. The results are placed into a folder called \textit{raw} inside \textit{csv\_data}
\end{description}